% Template taken from https://www.overleaf.com/latex/templates/kesmaths-beamer/szhrknpcspsz
% !TeX spellcheck = en_B
% !TeX encoding = UTF-8 %%%These comments just make you look like you know what you're doing
\documentclass[10pt]{beamer}  %%% Specifies the class, with 8pt font size

\usepackage[utf8]{inputenc}
\usetheme[block=fill,progressbar=foot,background=light]{metropolis}     
\usepackage[english]{babel}
\usepackage{csquotes}       
\usepackage[T1]{fontenc}        
\usepackage{booktabs}
\usepackage{pgfgantt}
\usepackage{pifont}
\usepackage{adfbullets}
\usepackage{enumitem}
\usepackage{amsmath}   
\usepackage{tikz}
\usepackage{lipsum}
\usepackage{amssymb}
\usepackage{amsfonts}
\usepackage{mathrsfs}   
\usepackage{graphicx}
\usepackage{adjustbox}
\usepackage{varioref}
\usepackage{probsoln}
\usepackage{attachfile2}
\usepackage{pgfplots}
\pgfplotsset{compat=newest}
\usepackage[]{hyperref} 
\graphicspath{{Graphics/}}
\usepackage{multirow,array}
\usepackage{colortbl}
\definecolor{aa}{RGB}{255, 124, 0}
\definecolor{cc}{RGB}{230, 230, 230}    
\usebackgroundtemplate{%
\tikz[overlay,remember picture]{\node[scale=80,opacity=0.03, at=(current page.south east)] {\adfbullet{9}};}}
  
\date[\today]{\today}

\usepackage{comment}
\usepackage{varwidth}

\newcommand{\mat}[4]{\left(\begin{array}{cc} #1 & #2 \\ #3 & #4 \\ \end{array}\right)}
\newcommand{\Q}{\mathbb{Q}}
\newcommand{\R}{\mathbb{R}}
\newcommand{\Z}{\mathbb{Z}}
\newcommand{\sol}[2][+]{
    \tikz[baseline]{\node[color=aa,fill=cc,rectangle,draw,anchor=base] {  {\onslide<#1->{#2}}  };}
}

\usetikzlibrary{positioning}
\usetikzlibrary{tikzmark}
\usetikzlibrary{shadings}
\usetikzlibrary{through}

\def\height{0.8cm}
\def\width{1.2cm}
\newcommand{\keynode}[6]{\node[minimum height=\height,minimum width=\width,draw,rectangle,color=aa,fill=cc] (#3) at (#1,#2) {};
    \node[rectangle,minimum height=\height/2,minimum width=\width,above,color=aa] at (#3) {#3};
    \node[draw,rectangle,minimum height=\height/2,minimum width=\width/3,below,color=aa,fill=cc,inner sep =0cm] at (#3) {\footnotesize#4};
    \node[draw,rectangle,minimum height=\height/2,minimum width=\width/3,below,xshift=\height/2,color=aa,fill=cc,inner sep=0cm] at (#3) {\footnotesize#5};
    \node[draw,rectangle,minimum height=\height/2,minimum width=\width/3,below,xshift=-\height/2,color=aa,fill=cc,inner sep=0cm] at (#3) {\footnotesize#6}; }

\newenvironment{gantt}[3]{\begin{ganttchart}[#1,bar height=.6,bar top shift=.2,title/.style=  {draw=none},y unit chart=0.6cm,y unit title = 0.6cm,include title in canvas=false,group/.append style={draw=black,dashed},bar/.append style={fill=aa},inline,hgrid=true,Float1/.style={bar/.append style={fill=none,dashed},bar height=.8,bar top shift=0.1}]{#2}{#3}}{\end{ganttchart}}

\newenvironment{nicetable}[1]{\setlength\arrayrulewidth{0.5mm}
            \arrayrulecolor{white}
            \begin{tabular}{#1}}{\end{tabular}}
        
\setlist[itemize,1]{label={\color{aa}\huge\adfbullet{9}}}
\setlist[itemize, 2]{label={\color{aa}\large\adfbullet{9}}}

\newcommand\reshist{}
\def\reshist(#1)#2(#3)#4(#5)%
{\draw (axis cs:#1) rectangle (axis cs:#3) node [midway] {#5};} %%%adds the general preamble

\title{\normalsize{Braess's Paradox - How Making Roads Could Slow Up Traffic}} %%%Title and subtitle
\subtitle{\normalsize{Param Rathour}} %%%This embeds the tikz source code into the resulting pdf...
% \subtitle{Template, \textattachfile{Braess Paradox.tex}{(TeX)}} %%%This embeds the tikz source code into the resulting pdf...

\begin{document}

\frame{\titlepage} %%%Makes titlepage

\setlength{\abovedisplayskip}{0pt}
\setlength{\belowdisplayskip}{0pt}
\setlength{\abovedisplayshortskip}{0pt}
\setlength{\belowdisplayshortskip}{0pt}  %%%Compresses math

\begin{frame}{What is Braess's Paradox}   %%%Frame with title
    This is an example of a Veridical Paradox.

    Adding capacity to a transportation network can sometimes actually slow down the traffic!
\end{frame}

\begin{frame}{Modelling a Transportation Network}   %%%Frame with title
    \begin{figure}
        \centering
        \includegraphics[width=0.3\linewidth]{1.pdf}
        \caption{A highway network}
    \end{figure}
    \begin{itemize}
        \item Directed Graph 
        \begin{description}
            \item[Edges] Highways
            \item[Nodes] Exits to get on or off a particular Highway.
        \end{description}
        \item Each edge has a designated travel time that depends on the amount of traffic it contains.
    \end{itemize}
\end{frame}

\begin{frame}{Strategic Form Games}   %%%Frame with title
    \begin{definition}[Strategic Form Game]
        A Strategic Form Game $\Gamma$ is a tuple $\langle N,(S_i)_{i\in N},(u_i)_{i\in N}\rangle$, where
        \begin{itemize}
            \item $N=\{1,2,\ldots,n\}$ is a set of players
            \item $S_1,S_2,\ldots,S_n$ are sets called the strategy sets of the players $1,2,\ldots,n$ respectively
            \item $u_i : S_1 \times S_2 \times \cdots \times S_n \rightarrow \R$ for $i = 1, 2,\ldots, n$ are mappings called the utility functions or payoff functions.
        \end{itemize}
    \end{definition}
\end{frame}

\begin{frame}{Representation into a Strategic Form Game}   %%%Frame with title
    \begin{figure}
        \centering
        \includegraphics[width=0.3\linewidth]{1.pdf}
        \caption{A highway network}
    \end{figure}
    \begin{itemize}
        \item Assume $n=4000$ cars, then $N=\{1,2,\ldots,4000\}$
        \item Strategy Sets are $S_1=S_2=\cdots=S_{4000}=\{C,D\}$
        \item Assume $n_C$ ($n_D$) cars travel along C (D), Note that $n_C+n_D=n$

        So, the utility functions are
        \begin{align*}
            u_i(s_1,\ldots,s_n)&=-45-\frac{n_C}{100} \quad \text{if} \quad s_i = C\\
            &=-45-\frac{n_D}{100} \quad \text{if} \quad s_i = D
        \end{align*}
    \end{itemize}
\end{frame}

\begin{frame}{The notion of Nash Equilibrium}   %%%Frame with title
    \begin{definition}[Pure Strategy Nash Equilibrium]
        Given a strategic form game $\Gamma = \langle N,(S_i),(u_i)\rangle$, the strategy profile $s^*=(s_1^*,s_2^*,\ldots,s_n^*)$ is called a pure strategy Nash equilibrium of $\Gamma$ if
        \[u_i(s^*_i,s^*_{-i})\geq u_i(s_i,s^*_{-i})\ \forall s_{i} \in S_{i}\ \forall i=1,2,\ldots,n\]
        That is, each player’s Nash equilibrium strategy is a best response to the Nash equilibrium strategies of the other players
    \end{definition}

    \begin{definition}[Best Response Correspondence]<2->
        Given a strategic form game $\Gamma = \langle N,(S_i),(u_i)\rangle$,the best response correspondence for player $i$ is the mapping $b_i : S_{-i} \rightarrow 2^{S_i}$ defined by
        \[b_i(s_{-i}) = \{s_i \in S_i : u_i(s_i, s{-i}) \geq u_i(s^\prime_i, s{-i}) \ \forall\  s^\prime_i \in S_i\}\]
        It can be seen that the strategy profile $(s_1^*,s_2^*,\ldots,s_n^*)$ is a pure strategy Nash equilibrium iff
        \[s^*_i \in b_i(s^*_{-i}), \forall i = 1,\ldots, n\]
    \end{definition}
\end{frame}

\begin{frame}{Interpretations of Nash Equilibrium}
    \begin{itemize}
        \item Prescription
        \item Prediction
        \item Self-Enforcing Agreement
        \item Evolution and Steady-State
    \end{itemize}
\end{frame}

\begin{frame}{Equilibrium Traffic}   %%%Frame with title
    \begin{figure}
        \centering
        \includegraphics[width=0.3\linewidth]{1.pdf}
        \caption{A highway network}
    \end{figure}
    \begin{itemize}
        \item First consider case when $n_C\neq n_D$, then the two routes will have unequal travel times, and any driver on the slower route would have an incentive to switch to the faster one. 
        \item Hence any list of strategies in which $n_C$ is not equal to 2000 cannot be a Nash equilibrium; and any list of strategies in which $n_C=n_D = 2000$ is a Nash equilibrium.
        \item Time delay = $45+\frac{2000}{100}=65$ minutes
    \end{itemize}
\end{frame}

\begin{frame}{Adding a Route from $C$ to $D$}
    \begin{figure}
        \centering
        \includegraphics[width=0.5\linewidth]{2.pdf}
        \caption{A highway network}
    \end{figure}
    \begin{itemize}
        \item Now, a fast link from $C$ to $D$ to ease the congestion in the network is introduced
        \item We will assume the travel time from $C$ to $D$ to be zero as a degenerate case
    \end{itemize}
\end{frame}

\begin{frame}{Representation into a Strategic Form Game}   %%%Frame with title
    \begin{figure}
        \centering
        \includegraphics[width=0.3\linewidth]{2.pdf}
        \caption{A highway network}
    \end{figure}
    \begin{itemize}
        \item Again, assume $n=4000$ cars, then $N=\{1,2,\ldots,4000\}$
        \item Strategy Sets are $S_1=S_2=\cdots=S_{4000}=\{C,D,CD\}$
        \item Assume $n_C$ ($n_D$) ($n_{CD}$) cars travel along $C$ ($D$) ($CD$), Note that $n_C+n_D+n_{CD}=n$

        So, the utility functions are
        \begin{align*}
            u_i(s_1,\ldots,s_n)&=-45-\frac{n_C+n_{CD}}{100} \quad \text{if} \quad s_i = C\\
            &=-45-\frac{n_D+n_{CD}}{100} \quad \text{if} \quad s_i = D\\
            &=-\frac{n_C+n_{CD}}{100}-\frac{n_D+n_{CD}}{100} \quad \text{if} \quad s_i = CD
        \end{align*}
    \end{itemize}
\end{frame}

\begin{frame}{Equilibrium Traffic}   %%%Frame with title
    \begin{figure}
        \centering
        \includegraphics[width=0.3\linewidth]{2.pdf}
        \caption{A highway network}
    \end{figure}
    \begin{itemize}
        \item A surprising result is that now there is a unique Nash equilibrium (every driver uses the route $CD$).
        \item Why is it an equilibrium?
        \item Why is it unique?
        \item Time delay = $\frac{4000}{100}+\frac{4000}{100}=80$ minutes
        \item This, time is clearly worse than 65 minutes we can get if half the people choose $C$ and other the half choose $D$
    \end{itemize}
\end{frame}

\begin{frame}{Big Questions}
    \begin{figure}
        \centering
        \includegraphics[width=0.5\linewidth]{5.pdf}
    \end{figure}
    \begin{itemize}
        \item Does an equilibrium traffic pattern always exists?
        \item How bad Braess's Paradox can be for networks in general?
        \item How much larger can the equilibrium travel time be after the addition of an edge, relative to what it was before?
        \item How to design networks to prevent bad equilibria from arising?
    \end{itemize}
\end{frame}

\begin{frame}{References}
    \begin{itemize}
        \item  \href{http://www.cs.cornell.edu/home/kleinber/networks-book/}{Chapter 8 \emph{Networks, Crowds, and Markets: Reasoning about a Highly Connected World} by David Easley and Jon Kleinberg, Cambridge University Press, 2010}
        \item \href{https://theory.stanford.edu/~tim/papers/routing.pdf}{\textbf{How Bad is Selfish Routing?} by Tim Roughgarden and Eva Tardos}
    \end{itemize}
\end{frame}
\end{document}