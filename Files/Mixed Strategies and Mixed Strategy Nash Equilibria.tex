\section{Mixed Strategies and Mixed Strategy Nash Equilibria}
\subsection{Mixed Strategies}
\begin{defn}[Mixed Strategy]
	Given a player $i$ with $S_i$ as the set of pure strategies, a mixed strategy (also called randomized strategy) $\sigma_i$ of player $i$ is a probability distribution over $S_i$.
	That is, $\sigma_i : S_i \rightarrow [0, 1]$ is a mapping that assigns to each pure strategy $s_i \in S_i$, a probability $\sigma_i(si)$ such that
	\[\sum_{s_i \in S_i}\sigma_i(s_i)=1\]
\end{defn}
A pure strategy of a player, say $s_i \in S_i$, can be considered as a mixed strategy that assigns probability 1 to $s_i$ and probability 0 to all other strategies of player $i$.
Such a mixed strategy is called a degenerate mixed strategy and is denoted by $e(s_i)$ or simply by $s_i$.
\begin{defn}[Mixed Extension]
	The set of all mixed strategies of player $i$ is the set of all probability distributions on the set $S_i$
	\[\Delta(S_i)=\left\{(\sigma_{i1},\ldots,\sigma_{im}) \in \R^m \ : \ \sigma_{ij}\geq 0 \ \text{for} \ j \in \{1,\ldots,m\}\ \text{and}\ \sum_{j=1}^m \sigma_{ij}=1\right\}\]
\end{defn}
Using the mixed extensions of strategy sets, we can define a mixed extension of the pure strategy game $\Gamma = \langle N,(S_i),(u_i)\rangle$ as 
\[\Gamma = \langle N,(\Delta(S_i)),(U_i)\rangle\]
\begin{note}
	$U_i$ is a mapping that maps mixed strategy profiles to real numbers
	\[U_i:\Delta(S_i)\times\cdots\times\Delta(S_n)\rightarrow\R\]
\end{note}
First, we make the standard assumption that the randomizations of individual players are mutually independent.
This implies that given a mixed strategy profile ($\sigma_1,\ldots,\sigma_n$), the random variables $\sigma_1,\ldots,\sigma_n$ are mutually independent.
Therefore the joint probability of a pure strategy profile ($s_1,\ldots,s_n$) is given by
\[\sigma(s_1,\ldots,s_n)=\prod_{i\in N}\sigma_i(s_i)\]
The payoff functions $U_i$ are defined as
\[U_i(\sigma_1,\ldots,\sigma_n)=\sum_{(s_1,\ldots,s_n)\in S} \sigma(s_1,\ldots,s_n)\cdot u_i(s_1,\ldots,s_n)\]
\subsection{Mixed Strategy Nash Equilibrium}
We now define the notion of a mixed strategy Nash equilibrium, which is a natural extension of the notion of pure strategy Nash equilibrium.
\begin{defn}[Mixed Strategy Nash Equilibrium]
Given a strategic form game $\Gamma = \langle N,(S_i),(u_i)\rangle$, the strategy profile $(\sigma_1^*,\sigma_2^*,\ldots,\sigma_n^*)$ is called a Nash equilibrium if $\forall i\in N$
\[u_i(\sigma^*_i,\sigma^*_{-i})\geq u_i(\sigma_i,\sigma^*_{-i})\ \forall \sigma_{i} \in \Delta(S_{i})\ \forall i=1,2,\ldots,n\]
Alternatively,
\[u_i(\sigma^*_i,\sigma^*_{-i})= \underbrace{\op{max}}_{s_{i} \in \Delta(S_{i})} u_i(s_i,\sigma^*_{-i})\ \forall i=1,2,\ldots,n\]
\end{defn}
Similar to \ref{def:brc}
\[b_i(\sigma{-i}) = \{\sigma_i \in \Delta(S_i) : u_i(\sigma_i, \sigma{-i}) \geq u_i(\sigma^\prime_i, \sigma{-i}) \ \forall\  \sigma^\prime_i \in \Delta(S_i)\}\]
That is, each player’s Nash equilibrium strategy is a best response to the Nash equilibrium strategies of the other players
\subsection{Properties of Mixed Strategies}
\begin{defn}[Convex Combination]
	Given real numbers $y_1,\ldots,y_n$, a convex combination of these numbers is a weighted sum of the form $\lambda_1y_1 + \lambda_2y_2 + \cdots + \lambda_ny_n$, where
	\[0\leq \lambda_i\leq1 \quad \text{for}\quad i \in \{1,\ldots,n\};\quad \sum_{i=1}^{n}\lambda_i=1\]
\end{defn}
\begin{prop}
	Let $\Gamma = \langle N,(S_i),(u_i)\rangle$ be a strategic form game.
	Then $u_i(\sigma_i, \sigma_{-i})$ can be expressed as the convex combination:
	\[u_i(\sigma_i,\sigma_{-i})=\sum_{s_i\in S_i} \sigma_i(s_i)\cdot u_i(s_i,\sigma_{-i})\]
	where
	\[u_i(s_i,\sigma_{-i})=\sum_{s_{-i}\in S_{-i}} \left(\prod_{j\neq i \sigma_j(s_j)}\right)u_i(s_i,s_{-i})\]
\end{prop}
\begin{prop}
	Given a strategic form game $\Gamma = \langle N,(S_i),(u_i)\rangle$, then, for any $\sigma\in \times_{i\in N}\Delta(S_i)$ and for any player $i\in N$
	\[\max_{\sigma_i\in\Delta(S_i)}u_i(\sigma_i,\sigma_{-i})=\max_{s_i\in S_i} u_{i}(s_i,\sigma_{-i})\]
	Furthermore,
	\[\rho_i\in \underbrace{\text{arg max}}_{\sigma_{i}\in \Delta(S_i)}u_i(\sigma_i,\sigma_{-i})\]
	iff
	\[\rho_i(x)=0 \ \forall x\notin \underbrace{\text{arg max}}_{\sigma_{i}\in S_i} u_i(s_i,\sigma_{-i})\]
\end{prop}
\subsection{Necessary and Sufficient Conditions for a Profile to be a Mixed Strategy Nash Equilibrium}
\begin{defn}[(Support of a Mixed Strategy]
	Let $\sigma_i$ be any mixed strategy of a player $i$.
	The support of $\sigma_i$, denoted by $\delta(\sigma_i)$, is the set of all pure strategies which have non-zero probabilities under $\sigma_i$, that is:
	\[\delta(\sigma_i)=\{s_i\in S_i:\sigma_i(s_i)>0\}\]
\end{defn}
\begin{defn}[Support of a Mixed Strategy Profile]
	Let $\sigma = (\sigma_1,\ldots,\sigma_n)$ be a mixed strategy profile with $\delta(\sigma_i)$ as the support of $\sigma_i$ for $i\in \{1,\ldots,n\}$.
	Then the support $\delta(\sigma)$ of the profile $\sigma$ is the Cartesian product of the individual supports, that is $\delta(\sigma_1) \times\ldots\times \delta(\sigma_n)$.
\end{defn}
\begin{theorem}
	The mixed strategy profile $(\sigma_1^*,\ldots,\sigma_n^*)$ is a mixed strategy Nash equilibrium iff $\forall i \in N$,
	\begin{itemize}
		\item $u_i(s_i,\sigma^*_{-1}$ is the same $\quad \forall s_i\in \delta(\sigma_i^*))$
		\item $u_i(s_i,\sigma^*_{-1} \geq u_i(s_i^\prime,\sigma^*_{-1}$ is the same $\quad \forall s_i\in \delta(\sigma_i^*));\quad \forall s_i^\prime \notin \delta(\sigma_i^*)$
	\end{itemize}
\end{theorem}
\subsubsection{Implications of the Necessary and Sufficient Conditions}
\begin{itemize}
	\item Given a mixed strategy Nash equilibrium, each player gets the same payoff (as in the equilibrium) by playing any pure strategy having positive probability in her equilibrium mixed strategy.
	\item The above implies that the player can be indifferent about which of the pure strategies (having positive probability in her equilibrium mixed strategy) she will play.
	Of course, when this player plays only one of these pure strategies, then it may not be a best response for the other players to play their Nash equilibrium strategies.
	\item To verify whether or not a mixed strategy profile is a Nash equilibrium, it is enough to consider the effects of only pure strategy deviations (with the rest of the players playing their equilibrium strategies).
\end{itemize}
\begin{prop}
	Given $s_i \in S_i$, let $e(s_i)$ denote the degenerate mixed strategy that assigns probability 1 to $s_i$ and probability 0 to all other strategies in $S_i$.
	The strategy profile $(s_1^*,\ldots,s_n^*)$ is a pure strategy Nash equilibrium of the game $\langle N,(S_i),(u_i)\rangle$ iff the mixed strategy profile $(e(s_1^*),\ldots,e(s_N^*))$ is a mixed strategy Nash equilibrium of the game $\langle N,(S_i),(u_i)\rangle$.
\end{prop}
\subsection{Maxmin Values and Minmax Values in Mixed Strategies}
These notions are similar to what discussed in Section \ref{sbs:mvmv}
\begin{defn}[Maxmin Value and Maxmin Strategy]
	\[\underline{v_i}=\max_{\sigma_i\in \Delta(S_i)} \min_{\sigma_{-i}\in \times_{j\neq i}\Delta(S_{-i})} u_i(\sigma_i,\sigma_{-i})\]
\end{defn}
\begin{defn}[Minmax Value and Minmax Strategy]
	\[\underline{v_i}=\min_{\sigma_{-i}\in \times_{j\neq i}\Delta(S_{-i})} \max_{\sigma_i\in \Delta(S_i)} u_i(\sigma_i,\sigma_{-i})\]
\end{defn}
The relations in Proposition \ref{prop:mvmvrne} still holds.
\subsection{Domination in Mixed Strategies}
\begin{defn}[Domination in Mixed Strategies]
	Given two mixed strategies $\sigma_i,\sigma_i^\prime\in \Delta(S_i)$ of player $i$,

	We say $\sigma_i$ strictly dominates $\sigma_i^\prime$ if
	\[u_i(\sigma_i,\sigma_{-i})>u_i(\sigma_i^\prime,\sigma_{-i}) \ \forall\sigma_{-i}\in \times_{j\neq i}\Delta(S_j)\]

	We say $\sigma_i$ weakly dominates $\sigma_i^\prime$ if
	\[u_i(\sigma_i,\sigma_{-i})\geq u_i(\sigma_i^\prime,\sigma_{-i}) \ \forall\sigma_{-i}\in \times_{j\neq i}\Delta(S_j)\]
	\[u_i(\sigma_i,\sigma_{-i})>u_i(\sigma_i^\prime,\sigma_{-i}) \ \text{for some} \sigma_{-i}\in \times_{j\neq i}\Delta(S_j)\]

	We say $\sigma_i$ very weakly dominates $\sigma_i^\prime$ if
	\[u_i(\sigma_i,\sigma_{-i})\geq u_i(\sigma_i^\prime,\sigma_{-i}) \ \forall\sigma_{-i}\in \times_{j\neq i}\Delta(S_j)\]
\end{defn}
\begin{defn}[Dominant Mixed Strategy Equilibrium]
	If the mixed strategy say $\sigma^*$ strongly (weakly) (very weakly) dominates all other strategies $\sigma_i^\prime\in\Delta(S_i)$, we say $\sigma_i^*$ is a strongly (weakly) (very weakly) dominant strategy of player $i$.

	A strategy profile $(\sigma_1^*,\ldots,\sigma_n^*)$ such that $\sigma^*_i$ is a strictly (weakly) (very weakly) dominant strategy for player $i$, $\forall i \in N$, is called a strictly (weakly) (very weakly) dominant mixed strategy equilibrium.
\end{defn}
\begin{note}
	Any dominant mixed strategy equilibrium is also a mixed strategy Nash equilibrium.
\end{note}
\begin{note}
	A strictly dominant mixed strategy for any player, if one exists, is unique.
\end{note}
\subsection{Iterated Elimination of Dominated Strategies}
We have observed that elimination of strictly dominated strategies simplifies analysis of games.
We shall formalize this as follows.
Consider a finite strategic form game $\langle N,(S_i),(u_i)\rangle$.
Let $k = 1, 2,\ldots,K$ denote the successive rounds in which strictly dominated strategies are eliminated.
For each player $i \in N$, define the sets of strategies $S_i^k$ as follows.
\begin{itemize}
	\item $S_i^1=S_i$
	\item $S_i^{k+1}\subseteq S_i^k$ for $k=1,2,\ldots,K-1$
	\item For $k=1,2,\ldots,K-1$, all strategies $s_i\in S_i^k\setminus S_i^{k+1}$ are strictly dominated strategies which are eliminated in the $k^{\text{th}}$ round from the game in which the set of strategies of $j\in N$ is $S_j^k$.
	\item No strategy in $S_i^K$ is strictly dominated in the game in which the set of strategies of each player $j\in N$ is $S_j^K$.
\end{itemize}
The above steps define the process of iterated elimination of strongly dominated strategies.
The set of strategy profiles
\[\{(s_1,s_2,\ldots,s_n):s_i\in S_i^K\ \text{for}\ i=1,\ldots,n\}\]
is said to survive the iterated elimination of strictly dominated strategies