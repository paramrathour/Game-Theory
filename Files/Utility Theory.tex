\section{Utility Theory}
Utilities play a central role in game theory.
They capture the preferences that the players have for different outcomes in terms of real numbers thus enabling realvalued functions to be used in game theoretic analysis.
So far we have implicitly assumed that utility functions can correctly and faithfully capture the preferences the players have for different outcomes.
The utility theory developed by von Neumann and Morgenstern provides a scientific justification for this assumption.
This section introduces and presents their axiomatic utility theory.
\subsection{Ordinal Utilities}
Let $\succeq$ represents the preference relation of player $i$, we are interested in a utility function $u_i (i = 1, 2)$ such that
\[x_1\succeq_i x_2 \Leftrightarrow u_i(x_1)\geq u_i(x_2)\]
A scale on which larger numbers represent more preferred outcomes in a way that only the order of the numbers matters and not their absolute or relative magnitude is called an \emph{ordinal scale}.
Utility numbers determined from preferences in this way are called \emph{ordinal utilities}.
\subsection{Preferences over Lotteries}
To describe the interaction of preferences when there is uncertainty about which outcome will be selected, the notion of a lottery (or probability distribution) is a natural tool that can be used.
Suppose $X = \{x_1, x_2,\ldots, x_m\}$.
Then a lottery on $X$ is a probability distribution
\[\sigma=[p_1:x_1;p_2:x_2;\ldots;p_m:x_m]\]
Note that
\[p_j\geq 0 \ \text{for}\ j=1,2,\ldots,m \ \text{and}\ \sum_{j=1}^m p_j=1\]
\subsubsection{Axioms of von Neumann - Morgenstern Utility Theory}
Let $X$ as usual denote the set of outcomes.
Consider a player $i$ and suppose we focus on the preferences that the player has over the outcomes in $X$.
These preferences can be expressed in the form of a binary relation $\succeq$ on $X$.
Given $x_1, x_2 \in X$, let us define the following for the given player $i$:
\begin{itemize}
	\item $x_1 \succeq x_2:$ outcome $x_1$ is weakly preferrred to outcome $x_2$
	\item $x_1 \succ x_2:$ outcome $x_1$ is strictly preferrred to outcome $x_2$
	\item $x_1 \sim x_2:$ outcome $x_1$ is equally preferrred to outcome $x_2$
\end{itemize}
It is clear that the relation $\succeq$ is reflexive.
\begin{ax}[Completeness]
	The completeness property means that every pair of outcomes is related by the preference relation.
	Moreover, the preference relation $\succeq$ induces an ordering on $X$ which allows for ties among outcomes.
	This can be formally expressed as
	\[x_1\succ x_2; \quad\text{or}\quad x_2\succ x_1; \quad\text{or}\quad x_1\sim x_2\quad  \forall x_1,x_2\in X\]
\end{ax}
\begin{ax}[Transitivity]
	This states that
	\[x_1\succeq x_2 \quad\text{and}\quad x_2\succeq x_3 \quad\rightarrow\quad x_1\succeq x_3\quad  \forall x_1,x_2,x_3\in X\]
\end{ax}
\begin{ax}[Substitutability]
	This axiom is often called \emph{independence}.
	If $x_1 \sim x_2$, then for all sequences of one or more outcomes $x_3,\ldots, x_m$, and sets of probabilities $p, p_3,\ldots, p_m$ such that
	\[p+\sum_{j=3}^m p_j=1\]
	the player is indifferent to the lotteries $\sigma_1 = [p : x_1; p_3 : x_3 ;\ldots; p_m : x_m]$ and $\sigma_2 = [p : x_2; p_3 : x_3 ;\ldots; p_m : x_m]$.
	We write this as $\sigma_1 \sim \sigma_2$ or
	\[[p : x_1; p_3 : x_3 ;\ldots; p_m : x_m] \sim [p : x_2; p_3 : x_3 ;\ldots; p_m : x_m]\]
\end{ax}
\begin{ax}[Decomposability]
	This axiom is often called \emph{simplification of lotteries}.
	Suppose $\sigma$ is a lottery over $X$ and let $P_\sigma(x_i)$ denote the probability that $x_i$ is selected by $\sigma$.
	The decomposability axiom states that
	\[P_{\sigma_1}(x_i)= P_{\sigma_2}(x_i)\ \forall x_i\in X \quad\rightarrow\quad \sigma_1\sim\sigma_2\ \forall\sigma_1,\sigma_2\in\Delta(X)\]
\end{ax}
\begin{ax}[Monotonicity]
	Consider a player who strictly prefers outcome $x_1$ to outcome $x_2$.
	Suppose $\sigma_1$ and $\sigma_2$ are two lotteries over $\{x_1, x_2\}$.
	Monotonicity implies that the player would prefer the lottery that assigns higher probability to $x_1$.
	More formally, $\forall x_1, x_2 \in X,$
	\[x_1\succ x_2 \ \text{and}\ 1\geq p > q\geq 0 \quad\rightarrow\quad [p:x_1;1-p:x_2]\succ[q:x_1;1-q:x_2]\]
	Intuitively, monotonicity means that players prefer more of a good thing.
\end{ax}
\begin{ax}[Continuity]
	This axiom states that $\forall x_1, x_2, x_3 \in X$,
	\[x_1\succ x_2 \ \text{and}\ x_2\succ x_3 \quad\rightarrow\quad \exists p\in[0,1]\ \text{such that}\ x_2 \sim[p:x_1;1-p;x_3]\]
	The implication of the above axiom is that any outcome $x_2$ such that outcome $x_1$ is strictly preferred to $x_2$ but outcome $x_2$ is strictly preferred to another outcome $x_3$ will be indifferent to a player with $[p : x1; 1 - p : x3]$ for some probability $p$.
\end{ax}
\subsection{The von Neumann - Morgenstern Theorem}
\begin{theorem}[The von Neumann - Morgenstern Theorem]
	Given a set of outcomes $X = \{x_1,\ldots, x_m\}$ and a preference relation $\succeq$ on $X$ that satisfies completeness, transitivity, substitutability, decomposability, monotonicity and continuity, there exists a utility function $u : X \rightarrow [0, 1]$ with the following two properties:
	\begin{itemize}
		\item $u(x_1)\geq u(x_2)$ iff $x_1\succeq x_2$, $\forall x_1,x_2\in X$
		\item $u([p_1:x_1;p_2:x_2;\ldots;p_m:x_m])=\sum_{j=1}^m p_j\cdot(x_j)$
	\end{itemize}
\end{theorem}
\begin{note}
	Note that the right hand side is linear in the probabilities $p1,\ldots, pm$.
	This is a noteworthy feature of the von Neumann - Morgenstern utility function.
	A utility function that satisfies above conditions is aptly called a \emph{von Neumann - Morgenstern utility function}.
\end{note}
\begin{prop}[Affine Transformation]
	Every positive affine transformation U(x) of a utility function u(x) that satisfies U(x) = au(x) + b, where a and b are constants and a > 0, yields another utility function (in this case U) that satisfies properties (1) and (2) of Theorem 8.
	.
\end{prop}
\begin{theorem}[Risk Attitudes]
	Suppose $x_1, x_2 \in \R$ represent any pair of monetary receipts by a player $i$.
	Then,
	\begin{itemize}
		\item Then player $i$ is risk neutral if $\forall p \in [0, 1]$,
		\[u_i([p:x_1;(1-p):x_2])=u_i([1:px_1+(1-p)x_2])\ \forall x_1,x_2\in \R\]
		\item Then player $i$ is risk averse if $\forall p \in [0, 1]$,
		\[u_i([p:x_1;(1-p):x_2])\leq u_i([1:px_1+(1-p)x_2])\ \forall x_1,x_2\in \R\]
		\item Then player $i$ is risk loving if $\forall p \in [0, 1]$,
		\[u_i([p:x_1;(1-p):x_2])\geq u_i([1:px_1+(1-p)x_2])\ \forall x_1,x_2\in \R\]
	\end{itemize}
\end{theorem}
\begin{note}
	The above theorem implies that the utility function of a risk averse player is concave while the utility function of a risk loving player is convex.
\end{note}