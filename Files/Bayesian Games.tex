\section{Bayesian Games}
We have so far studied strategic form games with complete information, where the the entire game is common knowledge to the players.
We will now study games with incomplete information, where at least one player has private information about the game which the other players may not know.
While complete information games provide a convenient and useful abstraction for strategic situations, incomplete information games are more realistic.
\subsection{Games with Incomplete Information}
A game with \emph{incomplete information} is one in which, when the players are ready to make a move, at least one player has \emph{private information} about the game which the other players may not know.
The initial private information that a player has, just before making a move in the game, is called the \emph{type} of the player.

For example, in an auction involving a single indivisible item, each player has a valuation for the item, and typically this player would know this valuation deterministically while the other players may only have probabilistic information about how much this player values the item.
\subsection{Strategic Form Game with Incomplete Information}
A Strategic Form Game with incomplete information $\Gamma$ is a tuple $\langle N,(\Theta_i),(S_i),(p_i),(u_i)\rangle$, where
\begin{itemize}
    \item $N=\{1,2,\ldots,n\}$ is a set of players
    \item $\Theta_i$ is the set of types of player $i$ where $i=1,2,\ldots,n$
    \item $S_i$ is the set of actions or pure strategies of player i where $i=1,2,\ldots,n$
    \item The belief function $p_i$ is a mapping from $\Theta_i$ into $\Delta(\Theta_{-i})$, the set of probability distributions over $\Theta_{-i}$.
    That is, for any possible type $\theta_i\in\Theta_{i}$, $p_i$ specifies a probability distribution $p_i(.
    \theta_i)$ over the set $\Theta_{-i}$ representing player $i$'s beliefs about the types of the other players if his own type were $\theta_i$;
    \item The payoff functions $u_i :\Theta_1 \times \Theta_2 \times \cdots \times \Theta_n \times S_1 \times S_2 \times \cdots \times S_n \rightarrow \R$ assigns to each profile of types and each profile of actions, a payoff that player $i$ would get.
\end{itemize}
When we study such a game, we assume that
\begin{itemize}
    \item Each player $i$ knows the entire structure of the game as defined above.
    \item Each player $i$ knows his own type $\theta_i\in\Theta_{i}$.
    The player learns his type through some signals and each element in his type set is a summary of the information gleaned from the signals.
    \item The above facts are common knowledge among all the players in $N$.
    \item The exact type of a player is not known deterministically to the other players who however have a probabilistic guess of what this type is.
    The belief functions $p_i$ describe these conditional probabilities.
    Note that the belief functions $p_i$ are also common knowledge among the players.
\end{itemize}
\begin{defn}[Consistency of Beliefs]
    We say beliefs $(p_i)i\in N$ are consistent if there is some common prior distribution over the set of type profiles $\Theta$ such that each player's beliefs given his type are just the conditional probability distributions that can be computed from the prior distribution.
\end{defn}
If the game is finite, beliefs are consistent if there exists some probability distribution $\mathbb{P} \in \Delta(\Theta)$ such that
\[p_i(\theta_{-i}|\theta_i)=\frac{{\mathbb{P}(\theta_i,\theta_{-i})}}{\sum_{t_{-i}\in\Theta_{-i}}{\mathbb{P}(\theta_i,t_{-i})}}\ \forall \theta_i\in\Theta_i; \theta_{-i}\in\Theta_{-i};\forall i\in N\]
\subsection{Type Agent Representation and the Selten Game}
This is a representation of Bayesian games that enables a Bayesian game to be transformed to a strategic form game (with complete information).
Given a Bayesian game $\langle N,(\Theta_i),(S_i),(p_i),(u_i)\rangle$ the Selten game is an equivalent strategic form game $\langle N^s,(S_{\theta_i})_{\theta_i\in\Theta_{i};i\in N},(U_{\theta_i})_{\theta_i\in\Theta_{i};i\in N}\rangle$.

The idea used in formulating a Selten game is to have type agents.
Each player in the original Bayesian game is now replaced with a number of type agents; in fact, a player is replaced by exactly as many type agents as the number of types in the type set of that player.
We can safely assume that the type sets of the players are mutually disjoint.
The set of players in the Selten game is given by:
\[N^s=\bigcup_{i\in N}\Theta_i\]
Note that each type agent of a particular player can play precisely the same actions as the player himself.
This means that for every $\theta_i\in\Theta_{i}$
\[S_{\theta_i}=S_i\]
The payoff function $U_{\theta_i}$ for each $\theta_i\in\Theta_{i}$ is the conditional expected utility to player $i$ in the Bayesian game given that $\theta_i$ is his actual type.
It is a mapping with the following domain and co-domain:
\[U_{\theta_i}:\left(\{\times,i\in N\},\{\times,\theta_i\in\Theta_{i},S_i\}\right)\rightarrow\R\]
\subsubsection{Payoff Computation in Selten Game}
From now on, when there is no confusion, we will use $u$ instead of $U$.
In general, given a Bayesian game $\Gamma = \langle N,(\Theta_i),(S_i),(p_i),(u_i)\rangle$, suppose $(s_1,\ldots, s_n)$ is a strategy profile where for $i = 1,\ldots, n$, $s_i$ is a mapping from $\theta_i$ to $S_i$.
Assume the current type of player $i$ to be $\theta_i$.
Then the expected utility to player $i$ is given by
\[u_{i}((s_i,s_{-i})|\theta_i)=\mathbb{E}_{\theta_{-i}}[(u_i(\theta_i,\theta_{-i},s_i(\theta_i),s_{-i}(\theta_{-i})))]\]
For a finite Bayesian game, the above immediately translates to
\[u_{i}((s_i,s_{-i})|\theta_i)=\sum_{t_{-i}\in\Theta_{-i}}p_i(t_{-i}|\theta_i)\cdot(u_i(\theta_i,\theta_{-i},s_i(\theta_i),s_{-i}(\theta_{-i})))\]
With this setup, we now define the notion of Bayesian Nash equilibrium.
\subsection{Bayesian Nash Equilibrium}
\begin{defn}[Pure Strategy Bayesian Nash Equilibrium]
    A pure strategy Bayesian Nash equilibrium in a Bayesian game $\Gamma = \langle N,(\Theta_i),(S_i),(p_i),(u_i)\rangle$ can be defined in a natural way as a pure strategy Nash equilibrium of the equivalent Selten game.
    That is, a profile of strategies $(s_1^*,\ldots,s_n^*)$ is a pure strategy Bayesian Nash equilibrium if $\forall i \in N; \forall s_i : \Theta_i \rightarrow S_i; \forall \theta_i \in \Theta_i$,
    \[u_{i}((s_i^*,s_{-i}^*)|\theta_i)\geq u_{i}((s_i,s_{-i}^*)|\theta_i)\]
    That is, $\forall i\in N; \forall a_i\in S_i; \forall \theta_i\in \Theta_{-i}$,
    \[\mathbb{E}_{\theta_{-i}}[u_{i}(\theta_i,\theta_{-i},s_i^*(\theta_i),s_{-i}^*(\theta_{-i}))]\geq\mathbb{E}_{\theta_{-i}}[u_{i}(\theta_i,\theta_{-i},a_i,s_{-i}^*(\theta_{-i}))]\]
\end{defn}
\subsection{Dominant Strategy Equilibria}
Dominant strategy equilibria of Bayesian games can again be defined using the Selten game representation.
\begin{defn}[Very Weakly Dominant Strategy Equilibrium]
    Given a Bayesian game, $\Gamma = \langle N,(\Theta_i),(S_i),(p_i),(u_i)\rangle$ a profile of strategies $(s_1^*,\ldots,s_n^*)$ is called a very weakly dominant strategy equilibrium if $\forall i\in N;\ \forall s_i\in \Theta_i;\ \forall s_{-i}\in \Theta_{-i},\ \forall \theta_i\in \Theta_i$.
    \[u_{i}((s_i^*,s_{-i})|\theta_i)\geq u_{i}((s_i,s_{-i})|\theta_i)\]
    That is, $\forall i\in N;\ \forall a_i\in S_i;\ \forall \theta_i\in \Theta_{i},\ \forall s_{-i}:\Theta_{-i}\rightarrow S_{-i}$
    \[u_{i}((s_i^*,s_{-i})|\theta_i)\geq u_{i}((s_i,s_{-i})|\theta_i)\]
\end{defn}